%% This is file `mcmthesis-demo.tex',
%% generated with the docstrip utility.
%%
%% The original source files were:
%%
%% mcmthesis.dtx  (with options: `demo')
%%
%% -----------------------------------
%%
%% This is a generated file.
%%
%% Copyright (C)
%%     2010 -- 2015 by Zhaoli
%%     2014 -- 2016 by Liam 
%%     2017 -- 2019 by Xuehan
%%
%% This work may be distributed and/or modified under the
%% conditions of the LaTeX Project Public License, either version 1.3
%% of this license or (at your option) any later version.
%%
%% This work has the LPPL maintenance status `maintained'.
%%
%% The Current Maintainer of this work is Xuehan.
%%
\documentclass{mcmthesis}
\mcmsetup{CTeX = false,   % 使用 CTeX 套装时,设置为 true
        tcn =2002121, problem = E,
        sheet = true, titleinsheet = true, keywordsinsheet = true,
        titlepage = true}
\usepackage{palatino}
\usepackage{mwe}
\usepackage{graphicx}
\usepackage{subcaption}
\usepackage{float}
\usepackage{multirow}
\usepackage{indentfirst}
\usepackage{gensymb}
\usepackage[ruled,lined,commentsnumbered]{algorithm2e}
\usepackage{geometry}
\usepackage{listings}
\usepackage{fontspec}
\usepackage{booktabs}
\usepackage{amsmath}
\usepackage{color}
\usepackage{framed}
\usepackage{xcolor}
\usepackage{fontspec}
\usepackage{mathrsfs}
\usepackage{fancyhdr}
\usepackage{diagbox}

\lstset{  
 	numbers=left,                % 在左侧显示行号
 	numberstyle=\tiny\color{gray},       % 设定行号格式
 	%frame=none,                             % 不显示背景边框
 	%backgroundcolor=\color[RGB]{245,245,244},% 设定背景颜色
 	keywordstyle=\color[RGB]{40,40,255},     % 设定关键字颜色
 	numberstyle=\footnotesize\color{darkgray},       
 	commentstyle=\it\color[RGB]{0,96,96},     % 设置代码注释的格式
 	basicstyle=\fontspec{Consolas},
 	stringstyle=\rmfamily\slshape\color[RGB]{128,0,0}, % 设置字符串格式
 	showstringspaces=false,            % 不显示字符串中的空格
 	language=Python,                      % 设置语言
}
\newtheorem{definition}{Definition}
\definecolor{shadecolor}{rgb}{0.95,0.95,0.95}

\begin{document}
\linespread{0.6} %%行间距
\setlength{\parskip}{0.5\baselineskip} %%段间距
\title{title}

\date{\today}
	\begin{abstract}
		% abstract	
		\begin{keywords}
		keyword1,keyword2,keyword3
		\end{keywords}
		
	\end{abstract}
\maketitle


\tableofcontents
\newpage
%以下为正文部分
\section{section 1}
	\subsection{1.1}
		\subsubsection{1.1.1}
		
			\begin{shaded}
				\begin{proof}
					
			
				\end{proof}
			\end{shaded}

\section{section 2}
	\subsection{2.1}
	\begin{algorithm}%伪代码
		\caption{the name of algorithm}\label{an_algorithm}
		\KwIn{input}
		\KwOut{output}
		%以下为算法内容
		$r\leftarrow t$\;
		$\Delta B^{\ast}\leftarrow -\infty$\;
		\While{$\Delta B\leq \Delta B^{\ast}$ and $r\leq T$}{
			$Q\leftarrow\arg\max_{Q\geq 0}\Delta B^{Q}_{t,r}(I_{t-1},B_{t-1})$\;
			$\Delta B\leftarrow \Delta B^{Q}_{t,r}(I_{t-1},B_{t-1})/(r-t+1)$\;
			\If{$\Delta B\geq \Delta B^{\ast}$}{
				$Q^{\ast}\leftarrow Q$\;
				$\Delta B^{\ast}\leftarrow \Delta B$\;
			}
			$r\leftarrow r+1$\;
		}
	\end{algorithm}
	111
	\cite{WWFReport}
	\begin{table}[h]%表格
		\centering
		\caption{tablename}\label{table}
		\begin{tabular}{cccccccc}
			\toprule
			\\
			\midrule
			\\
			\\
			\bottomrule
		\end{tabular}
	\end{table}

\bibliographystyle{IEEEtran}
\bibliography{ref}

\newpage
\begin{appendices}

algorithm1
\begin{lstlisting}%源代码	

	print("hello world") 
\end{lstlisting}


\end{appendices}
\end{document}